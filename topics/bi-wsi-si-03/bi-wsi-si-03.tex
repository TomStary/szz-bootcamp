\documentclass{szzclass}
\usepackage{hyperref}
\usepackage{longtable}
\usepackage{booktabs}

\subject{DBS}
\code{BI-WSI-SI-03}
\topic{Normalizace relačního schématu a normální formy (první, druhá, třetí, BCNF).}
\providecommand{\tightlist}{%
  \setlength{\itemsep}{0pt}\setlength{\parskip}{0pt}}

\begin{document}
\tableofcontents
\newpage

\section{Normalizace}
Cílem normalizace je minimalizovat redundanci data a aktualizační anomálie. Měla by vést k vzniku tabulek,
které lze snadno udržovat.

\subsection{Fuknční závislost}
$X$ je množina všech atributů, $Y$ a $Z$ jsou $\subseteq$ $X$. $Z$ závisí funkčně na $Y$, jestliže ke každé $Y$ hodnotě
existuje maximálně jedna $Z$ hodnota. Značí se jako $Y \rightarrow Z$.

\subsubsection{Příklad}
\begin{itemize}
  \item ke každému kinu existuje nanejvýš jedna adresa $\Rightarrow$ nazev\_kina $\rightarrow$ adresa
  \begin{itemize}
    \item název kina určuje atribut adresa
    \item naopak adresa závizí na názvu kina
  \end{itemize}
  \item pro každou dvojici {kino, film} existuje nejvýše jedno datum, kdy se film v kině vysílá $\Rightarrow$ nazev\_kina, jmeno\_filmu $\rightarrow$ datum
  \begin{itemize}
    \item dvojice jméno kina a jméno filmu určuje atribut datum
    \item datum závisí na dvojici jmen
  \end{itemize}
\end{itemize}
\subsubsection{Amstrongova pravidla}
\begin{itemize}
  \item triviální funkční závislost = jestliže Y je $\subseteq$ X, pak X $\rightarrow$ Y
  \item tranzitivita = jestliže X $\rightarrow$ Y a Y $\rightarrow$ Z, pak X $\rightarrow$ Z
  \item kompozice pravé strany = X $\rightarrow$ Y a X $\rightarrow$ Z, pak X $\rightarrow$ YZ
  \item dekompozice pravé strany = X $\rightarrow$ YZ, pak X $\rightarrow$ Y a X $\rightarrow$ Z
\end{itemize}
\subsubsection{Klíč relace}
Řekněme R(A), kde K je $\subseteq$ A. Tak K je klíčem schématu R(A), jestliže splňuje dvě vlastnosti:
\begin{itemize}
  \item K $\rightarrow$ A
  \item neexistuje K' $\subset$ K taková, že K' $\rightarrow$ A
\end{itemize}

\section{Normální formy}
\begin{itemize}
  \item 1NF - atributy jsou atomické (nexistují strukturované a vicehodnotové atributy)
  \item 2NF - žádný neklíčový atribut není závislý na části klíče (vždy závisí na celém)
  \item 3NF - jsou-li všechny neklíčové atributy závislé na klíči přímo a nikoliv tranzitivně
  \begin{itemize}
    \item nesmí tranzitivně záviset na jiném klíči
    \item nesmí záviset na jiném neklíčovém atributu
    \item musí záviset pouze na primárním klíči
  \end{itemize}
\end{itemize}
\subsection{BCNF - Boyce-Coddova normální forma}
Schéma relace R je v BNFC, jestliže pro každou netriviální závistlo X $\rightarrow$ Y platí, že X obsahuje klíč schématu R. Pokud je schéma v BCNF,
pak je i v 3NF.

\section{Normalizace}
Eliminace aktualizačních anomálií se zajišťuje převedením relačního schématu do 3NF (resp. BCNF). Lze normalizovat pomocí dekompozice.
Požadavky dekompozice:
\begin{itemize}
  \item výsledná schémata by měla mít "stejnou" sémantiku - nové i staré schéma musí mít stejnou funkční závislost
  \item nová relace by měla obsahovat "stejná" data, jaká by obsahovala původní relace
  \item o dekompozici lze uvažovat jako o projekcích původní relace, kde každá projekce definujé nové relace, a jejich spojením se musí dostat
  stejná data jako měla původní relace - \texttt{beztrátová dekompozice}
\end{itemize}
\subsection{Beztrátová dekompozice}
Musí být splněna pravidla:
\begin{itemize}
  \item schéma R(\underline{A},B,C), kde A, B, C jsou disjuktní, a je splněna funkční závislost B $\rightarrow$ C
  \item po rozdělení R na schémata R1(\underline{B},C) a R2(\underline{A},B) $\Leftrightarrow$ je rozdělení (dekompozice) bezztrátové
\end{itemize}

Normalizace je důležitá u "write intesive" databází, databáze kde se hodně zapisují a upravují data.
\newline
Naopak u "read only" databází se vyplatí denormalizace, zavedením redundance se sníží odezva na složitejší dotazy, protože denormalizované
databáze mají většinou méně tabulek než normalizované. A každé spojování tabulek při dotazování prodlužuje dobu odezvy.
\end{document}