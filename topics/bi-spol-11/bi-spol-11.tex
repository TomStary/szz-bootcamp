\documentclass{szzclass}
\usepackage[czech]{babel}

\subject{DBS}
\code{BI-SPOL-11}
\topic{3 úrovně pohledu na data (konceptuální, implementační, fyzická). Struktury pro ukládání dat v relačních databázích s ohledem na rychlý přístup k nim (speciální způsoby uložení, indexy apod.)}

\begin{document}
\section{3 úrovně pohledu na data}

\begin{description}
  \item[Konceptuální] Modelování reality (Obvykle se zachycuje se UML diagramem nebo ER modelem). Snaží se nebýt ovlivněna prostředky řešení.
  \item[Implementační] Konkrétní databázový model, konstrukční dotazovací a manipulační prostředky (relační, objektová, síťová, hierarchická, XML, \dots)
  \item[Fyzická] Sekvenční soubory, indexy, clustery apod.
\end{description}

\section{Struktury pro ukládání dat v relačních DB s ohledem na rychlý přístup k nim (speciální způsoby uložení, indexy apod.)}

\end{document}
