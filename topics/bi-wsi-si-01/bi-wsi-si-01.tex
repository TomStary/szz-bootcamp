\documentclass{szzclass}

\title{Informační bezpečnost, architektura bezpečnosti v modelu OSI.}
\author{Tomáš Starý}

\begin{document}

\section{Komponenty informační bezpečnosti}

Komponenty lze rozdělit takto:
\begin{enumerate}
    \item Informační technologie (IT): Technologie na zpracování informací.
    \item Informační a komunikační technologie (ICT): technologie zahrnující jak počítačové systémy
    tak také telekomunikační sítě pro zpracovaní informací.
\end{enumerate}

Kvůli zpracovaní a přenosu informací je potřeba řešit bezpečnost IT a ICT systémů. Bezpečnost dělíme na 2 úrovně:
\begin{enumerate}
    \item Počítačová bezpečnost (computer security)
          Představuje souhrn prostředků zabezpečující bezpečný provoz počítačů a ochranu dat
          zpracovaných a uchovávaných na počítači.
    \item Síťová bezpečnost (network security)
          Představuje souhrn prostředků zabezpečující ochranu dat po dobu jeich přenosu komunikačním
          prostředím a ochranu počítačů projených do počítačové sítě.
\end{enumerate}

Hranice mezi tímto dělením ovšem není jednoznačná, nebo

\end{document}