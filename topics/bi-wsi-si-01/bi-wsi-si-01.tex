\documentclass{szzclass}

\title{Informační bezpečnost, architektura bezpečnosti v modelu OSI.}
\author{Tomáš Starý}

\begin{document}

\maketitle
\tableofcontents
\newpage

\section{Komponenty informační bezpečnosti}

Komponenty lze rozdělit takto:
\begin{enumerate}
    \item Informační technologie (IT): Technologie na zpracování informací.
    \item Informační a komunikační technologie (ICT): technologie zahrnující jak počítačové systémy
    tak také telekomunikační sítě pro zpracovaní informací.
\end{enumerate}

Kvůli zpracovaní a přenosu informací je potřeba řešit bezpečnost IT a ICT systémů. Bezpečnost dělíme na 2 úrovně:
\begin{enumerate}
    \item Počítačová bezpečnost (computer security)
          Představuje souhrn prostředků zabezpečující bezpečný provoz počítačů a ochranu dat
          zpracovaných a uchovávaných na počítači.
    \item Síťová bezpečnost (network security)
          Představuje souhrn prostředků zabezpečující ochranu dat po dobu jeich přenosu komunikačním
          prostředím a ochranu počítačů projených do počítačové sítě.
\end{enumerate}

Hranice mezi tímto dělením ovšem není jednoznačná, například počítačové viry míří na počítačové systémy, ale
mohou narušit celou počítačovou síť.

\section{Informační bezpečnost}

Definice: Souhrn prostředků a postupů na zabezpečení důvěrnosti, integrity a dostupnosti informací.

Tyto prostředky mají za učel definovat postupy pro ochranu uživatelských dat, také řízení uživatelského
přistupu, tak aby pouze ověření uživatelé měli přístup k datům a to pouze k datům, kte kterým mají mít přístup.

\begin{itemize}
    \item Důvěrnost (confidentiality)
          Vlastnost, která zaručuje ochranu dat před neautorizovaným přístupem.
    \item Integrita (integrity)
          Zaručuje ucelenost a nepoškozenost informace.
    \item Dostupnost (availability)
          Dostupnost zaručuje přístup k informacím pro autorizované subjekty.
\end{itemize}

U informačního systému (IS) je potřeba myslet vždy na systém jako na celek a celá jeho
bezpečnost je stejně dobrá jako jeho nejslabší článek.

\subsection{Užitková hodnota (asset)}

Každý informační systém se skládá z užitkovž hodnot, což je vše co představuje určitou hodnotu pro organizaci, firmu nebo kterýkoliv ekonomický subjekt.

Tyto hodnoty dělíme do tří kategorií:
\begin{enumerate}
    \item Hmotné užitkové hodnoty (physical assets):
          Jedná se hlavně o technické prostředky jako například počítače a další hardware, ale také budovy.
    \item Nehmotné užitkové hodnoty (nonphysical assets):
          Software, informace (data) a možnost poskytování informací a služeb.
    \item Lidské zdroje (human resources):
          Personál s přítupem k informacím či správci IS.
\end{enumerate}

Užitkové hodnoty jsou vystavené různým hrozbám (threats), které představují potenciální narušení bezpečnosti IS a možnému úniku dat, jejich ztráty
či poškození.

\newpage

Hrozby můžeme taky rozdělit na dva typy a to úmyslná, která jsou prováděna za účelem získání dat nebo jejich poškození. Poté je zde neúmyslná hrozba,
která pramení z možnosti že uživatel udělá chybu, například nesprávným zacházením s IS. Hrozby také mohou být přírodního charakteru.
U charakteristiky hrozeb musíme nahlížet na:
\begin{itemize}
    \item zdroj
    \item motivaci
    \item možnou početnost výskytů
    \item jejich pravděpodobnost
    \item a následné dopady
\end{itemize}

Hrozby z hlediska IS:
\begin{itemize}
    \item Softwarová hrozba
          Jedná se hlavně o počítačové malware, které se zaměřují na napedení komunikace mezi jednotlivými částmi IS. Do systému
          se mohou dostat skrze paměťová média nebo přes síť, do které je daná komponenta připojená.
    \item Hrozba neautorizovaným subjektem (útočníkem)
          Potencionální možnost pro neověřeného uživatele dostat se do našeho systému za použití jiných nástrojů než již zmíněného malwaru.
\end{itemize}

Hrozby je tedy třeba ohodnit a na základě nich definovat jednotlivé zranitelnosti (vulnerability), které ukazují na slabá místa v IS a na
základě, kterých můžeme určit jejich dopady a také možná opatření jak jim předejít. Zranitelnost můžeme stejně jako hrozbu ohodnotit
(například: nízká, střední a vysoká). Samotná zranitelnost nezpůsobuje žádné škody pouze vytváří pro ně vytváří podmínky.

Při neočekávané či nežádoucí situaci mluvíme o incidentu infomační bezpečnosti. V návaznosti na incidenty můžeme mluvit o jejich dopadu (impact).
Dopad může být but přímý (ztráta data či integrity IS) nebo nepřímý (finanční ztráty, ztráta konkurence schopnosti). Posouzení dopadu je pak založeno
na poměru mezi cenou za ztrátu či narušené bezpečnosti a nástrojům, které zajistí, aby se situace neopakovala.



\end{document}