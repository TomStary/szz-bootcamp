\documentclass{szzclass}

\usepackage{amsmath}
\usepackage{graphics}
\usepackage[export]{adjustbox}[2011/08/13]

% \usepackage[czech]{babel}
% \usepackage[margin=3cm]{geometry}
% \usepackage{wrapfig}

% spacing
\usepackage{titlesec}
% \titlespacing*{\section}{0pt}{1ex}{0.5ex}
\titlespacing*{\subsection}{0pt}{1ex}{0ex}
\usepackage{dependencies/szz-math}

\topic{Číselné řady (konvergence číselné řady, kritéria konvergence, odhadování rychlosti růstu řad pomocí určitého integrálu).}
\author{Daniel Hampl}
\code{BI-SPOL-36}
\subject{ZMA}

\begin{document}
% \maketitle

\tableofcontents
\newpage


\section{Posloupnost}
\subsection{Definice}
Zobrazení množiny $\mathbb{N}$ do množiny $\mathbb{R}$ nazýváme reálná posloupnost.

\subsection{Limita}
Reálná posloupnost $(a_n)_{n=1}^\infty$ má limitu $\alpha\in\overline{\mathbb{R}}$,
právě když pro každé okolí $H_\alpha$ bodu $\alpha$ lze nalézt $n_0\in\mathbb{N}$ takové,
že pro všechna $n\in\mathbb{N}$ větší než $n_0$ platí $a_n\in H_\alpha$. V symbolech

\begin{equation*}
\big({\forall H_\alpha} \big) \big({\exists n_0\in\mathbb{N}} \big) \big({\forall n \in \mathbb{N}} \big) \big({n>n_0 \, \Rightarrow \, a_n \in H_\alpha} \big).\end{equation*}

Tuto skutečnost můžeme zapsat několika možnými ekvivalentními způsoby:

\begin{equation*}
\lim_{n\to\infty} a_n = \alpha \quad \text{nebo} \quad \lim a_n = \alpha \quad \text{nebo} \quad a_n \to \alpha.\end{equation*}

\subsection{Konvergence}
Buď $(a_n)_{n=1}^\infty$ posloupnost. Pokud pro její limitu platí
$\displaystyle\lim_{n\to\infty} a_n \in\mathbb{R}$,
pak se nazývá konvergentní. V ostatních případech ji nazýváme divergentní.

\section{Řada}
\subsection{Definice}
Formální výraz tvaru

\begin{equation*}
\sum_{k=0}^\infty a_k = a_0 + a_1 + a_2 + \cdots,\end{equation*}

kde $(a_k)_{k=0}^\infty$ je zadaná číselná posloupnost,
nazýváme číselnou řadou. Pokud je posloupnost částečných součtů

\begin{equation*}
s_n := \sum_{k=0}^n a_k, \quad n\in\mathbb{N}_0,\end{equation*}

konvergentní, nazýváme příslušnou řadu také konvergentní.
V opačném případě mluvíme o divergentní číselné řadě.
Součtem konvergentní řady $\sum_{k=0}^\infty a_k$
nazýváme hodnotu limity $\displaystyle\lim_{n\to\infty} s_n$


\subsection{Konvergence řady}
\subsubsection{Nutná podmínka konvergence}
Pokud řada $\sum_{k=0}^\infty a_k$ konverguje,
potom pro limitu jejích sčítanců platí
$\displaystyle \lim_{k\to\infty} a_k = 0$.

\textbf{Důsledek}
Pokud limita posloupnosti $(a_k)^∞_k=0$ je
nenulová nebo neexistuje, potom řada
$\sum_{k=0}^\infty a_k$ není konvergentní.

\subsubsection{Bolzano-Cauchy}
Řada $\displaystyle\sum_{k=0}^\infty a_k$
konverguje právě tehdy, když pro každé
$\epsilon>0$ existuje $n_0\in\textbf{R}$
tak, že pro každé $n\geq n_0$ a $p\in\mathbb{N}$ platí

\begin{equation*}
|a_n + a_{n+1} + \cdots + a_{n+p}| < \epsilon .\end{equation*}

\subsubsection{Absolutní konvergence}

Číselnou řadu $\sum_{k=0}^\infty a_k$ nazýváme absolutně
konvergentní, pokud číselná řada $\sum_{k=0}^\infty |a_k|$ konverguje.

Pokud řada absolutně konverguje, potom tato řada konverguje.

\subsubsection{Leibnizovo kritérium}
Buď $(a_k)_{k=0}^\infty$ klesající posloupnost s nezápornými členy konvergující k nule. Potom je řada

\begin{equation*}
\sum_{k=0}^\infty (-1)^k a_k\end{equation*}

konvergentní.

\subsubsection{Srovnávací kritérium}
Buďte $\sum_{k=0}^\infty a_k$ a $\sum_{k=0}^\infty b_k$
číselné řady. Potom platí následující dvě tvrzení.
\begin{itemize}
    \item Nechť pro každé $k\in\mathbb{N}$ platí nerovnost
    $0\leq|a_k|\leq b_k$ a nechť řada $\sum_{k=0}^\infty a_k$
    konverguje. Potom řada $\sum_{k=0}^\infty b_k$absolutně konverguje.
    \item Nechť pro každé $k\in\mathbb{N}$ platí nerovnosti
    $0\leq a_k\leq b_k$ a $\sum_{k=0}^\infty a_k$ diverguje.
    Potom i řada $\sum_{k=0}^\infty b_k$ diverguje.
\end{itemize}

\subsubsection{d'Alembertovo kritérium}
Nechť $a_k>0$ pro každé $k\in \mathbb{N}_0$. Pokud

\begin{equation*}
\lim_{k\to\infty} \frac{a_{k+1}}{a_k} > 1,\end{equation*}

potom řada  $\sum_{k=0}^\infty a_k$ diverguje. Pokud ovšem

\begin{equation*}
\lim_{k\to\infty} \frac{a_{k+1}}{a_k} < 1,\end{equation*}

potom řada  $\sum_{k=0}^\infty a_k$ konverguje.

\subsection{Odhadování růstu}
Nechť $f$ je spojitá funkce na $\langle 1,+\infty)$ a $n\in\mathbb{N}$. Je-li $f$ klesající, pak

\begin{equation*}
f(n) + \int_1^n f(x) \,\mathrm{d} x \leq \sum_{k=1}^n f(k) \leq f(1) + \int_1^n f(x) \,\mathrm{d}x.\end{equation*}

Je-li $f$ rostoucí, pak

\begin{equation*}
f(1) + \int_1^n f(x) \,\mathrm{d} x \leq \sum_{k=1}^n f(k) \leq f(n) + \int_1^n f(x) \,\mathrm{d}x.\end{equation*}

\subsubsection{Integráln kritérium}

Buď $\displaystyle\sum_{n=1}^\infty a_n$ číselná řada s kladnými členy taková,
že existuje spojitá a monotónní funkce definovaná na $\langle 1,+\infty)$ taková,
že $f(n)=a_n$ pro každé $n$. Potom
\begin{itemize}
    \item Pokud integrál $\displaystyle\int_1^\infty f(x)\,\mathrm{d}x$ konverguje,
    pak číselná řada $\displaystyle\sum_{n=1}^\infty a_n$ konverguje.
    \item Pokud integrál $\displaystyle\int_1^\infty f(x)\,\mathrm{d}x$ diverguje,
    pak číselná řada $\displaystyle\sum_{n=1}^\infty a_n$ diverguje.
\end{itemize}






\end{document}