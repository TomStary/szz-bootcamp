\documentclass{szzclass}
\usepackage{hyperref}
\usepackage{longtable}
\usepackage{booktabs}

\subject{PSI}
\code{BI-SPOL-24}
\topic{Protokolová rodina TCP/IP (IPv4, IPv6, TCP, UDP, aplikační
protokoly). Řízení datového toku. Princip a využití NAT. Systém DNS.}

\providecommand{\tightlist}{%
  \setlength{\itemsep}{0pt}\setlength{\parskip}{0pt}}

\begin{document}


% \hypertarget{spol-24-psi}{%
% \section{SPOL-24-PSI}\label{spol-24-psi}}

% \emph{Protokolová rodina TCP/IP (IPv4, IPv6, TCP, UDP, aplikační
% protokoly). Řízení datového toku. Princip a využití NAT. Systém DNS.}

\hypertarget{protokolovuxe1-rodina-tcpip}{%
\section{Protokolová rodina
TCP/IP}\label{protokolovuxe1-rodina-tcpip}}

\hypertarget{ipv4}{%
\subsection{IPv4}\label{ipv4}}

\begin{itemize}
\tightlist
\item
  32bit adresy
\item
  privátní rozsahy adres (neroutují se do internetu):

  \begin{itemize}
  \tightlist
  \item
    10.0.0.0/8
  \item
    172.16.0.0/12
  \item
    192.168.0.0/16
  \end{itemize}
\item
  MTU (Maximum Transmission Unit) - maximální délka rámce

  \begin{itemize}
  \tightlist
  \item
    definováno linkovou vrstvou
  \item
    typicky 1500 bytů (vyšší redukuje overhead, nižší transportní
    zpoždění)
  \item
    každý router může fragmentovat paket - sestavení až v cílovém
    zařízení
  \end{itemize}
\end{itemize}

\hypertarget{ipv6}{%
\subsection{IPv6}\label{ipv6}}

\begin{itemize}
\tightlist
\item
  128bit adresy
\item
  Hop limit - obdoba TTL u IPv4
\item
  minimální MTU je 1280 bytů
\item
  pokud je paket moc dlouhý, tak ho router zahodí a odešle ICMP zprávu s
  informací o MTU
\item
  typy adres:

  \begin{itemize}
  \tightlist
  \item
    unicast (individuální)
  \item
    multicast (skupinové)
  \item
    anycast (výběrové)
  \end{itemize}
\item
  adresní prostor:

  \begin{itemize}
  \tightlist
  \item
    ::1/128 loopback
  \item
    fc00::/7 individuální lokální adresy (obdoba privátních u IPv4)
  \item
    fe80::/10 lokální linkové adresy
  \item
    ff00::/8 skupinové adresy (multicast)
  \item
    2001:db8::/32 dokumentační příklady
  \end{itemize}
\item
  síťová rozhraní mají více adres
\end{itemize}

\hypertarget{tcp-transmission-control-protocol}{%
\subsection{TCP (Transmission Control
Protocol)}\label{tcp-transmission-control-protocol}}

\begin{itemize}
\tightlist
\item
  služba v transportní vrstvě (ISO/OSI)
\item
  spojově orientovaná, duplexní, v jedné relaci lze přenášet neomezeně
  dat
\item
  zabezpečení

  \begin{itemize}
  \tightlist
  \item
    kontrolní součty
  \item
    detekce duplicitních paketů
  \item
    správné seřazení
  \item
    opakované odeslání a timeout
  \end{itemize}
\item
  zahájení spojení - třícestný handshake (SYN, SYN+ACK, ACK)
\item
  ukončení spojení - (FIN, ACK, FIN, ACK)
\item
  nevhodné pro real-time aplikace (streaming, \ldots), vestavné systémy
  (příliš komplexní), \ldots{}
\end{itemize}

\hypertarget{udp-user-datagram-protocol}{%
\subsection{UDP (User Datagram
Protocol)}\label{udp-user-datagram-protocol}}

\begin{itemize}
\tightlist
\item
  služba v transportní vrstvě (ISO/OSI)
\item
  nespojová, nezabezpečená
\item
  výhodné kde vadí režie TCP - malé bloky dat, nevadí ztráta, real-time
  aplikace
\end{itemize}

\hypertarget{aplikaux10dnuxed-protokoly-sluux17eby}{%
\subsection{Aplikační protokoly
(služby)}\label{aplikaux10dnuxed-protokoly-sluux17eby}}

\begin{itemize}
\tightlist
\item
  využívají služeb transportní vrstvy (TCP/IP model), nebo prezentační
  vrstvy (ISO/OSI)
\item
  server nabízí službu, klient se připojí a službu využívá (alternativa
  P2P, kde se strany nerozlišují)
\end{itemize}

\hypertarget{dns}{%
\paragraph{DNS}\label{dns}}

\begin{itemize}
\tightlist
\item
  rozebírán v další části otázky
\end{itemize}

\hypertarget{ftp}{%
\paragraph{FTP}\label{ftp}}

\begin{itemize}
\tightlist
\item
  příkazový kanál port 21/TCP
\item
  datový kanál dynamicky přidělený port (také TCP) - aktivní/pasivní
\end{itemize}

\hypertarget{telnet}{%
\paragraph{Telnet}\label{telnet}}

\begin{itemize}
\tightlist
\item
  interaktivní příkazový terminál
\item
  port 23/TCP
\item
  nepodporuje šifrování (NEBEZPEČNÉ!)
\end{itemize}

\hypertarget{ssh}{%
\paragraph{SSH}\label{ssh}}

\begin{itemize}
\tightlist
\item
  port 22/TCP
\item
  náhrada Telnetu s šifrováním
\end{itemize}

\hypertarget{mail}{%
\subsection{Mail}\label{mail}}

\begin{itemize}
\tightlist
\item
  skupina protokolů: SMTP, IMAP4, POP3
\end{itemize}

\hypertarget{https}{%
\paragraph{HTTP(S)}\label{https}}

\begin{itemize}
\tightlist
\item
  80(443)/TCP
\end{itemize}

\hypertarget{dhcp---dynamic-host-configuration-protocol}{%
\paragraph{DHCP - Dynamic Host Configuration
Protocol}\label{dhcp---dynamic-host-configuration-protocol}}

\begin{itemize}
\tightlist
\item
  umožní klientovy získat konfiguraci (adresu, GW, \ldots)
\end{itemize}

\hypertarget{ux159uxedzenuxed-datovuxe9ho-toku}{%
\section{Řízení datového
toku}\label{ux159uxedzenuxed-datovuxe9ho-toku}}
\subsection{Řízení datového toku - flow control}
\begin{itemize}
  \item kontroluje se mezi jedním senderem a reciverem
  \item "plovoucí okénko" (slinding window)
  \item stop-and-wait (ACK)
  \item může se přímo říct odesílateli rychlost kterou by měl odesílat
\end{itemize}
\subsection{Kontrola zahlcení (congestion control)}
Detekce pomocé packet loss nebo zvětšení zpoždění
\begin{itemize}
  \item traffic shaping (Token bucket, Leaky bucket)
  \item rezervace pásma pro určitě spoje
\end{itemize}

\hypertarget{princip-a-vyuux17eituxed-nat-network-adress-translation}{%
\section{Princip a využití NAT (Network Adress
Translation)}\label{princip-a-vyuux17eituxed-nat-network-adress-translation}}

\begin{itemize}
\tightlist
\item
  překlad síťových adres
\item
  umožňuje připojit více počítačů na jednu veřejnou IP (obchází problém
  s nedostatkem IPv4 adres)
\item
  přepisuje port, adresu nebo jinou hodnotu v paketu
\item 
  striktně odděluje LAN od WAN
\item 
  funguje jako směrovač (router)
\item
  druhy:

  \begin{itemize}
  \tightlist
  \item
    Source - změna zdrojového portu nebo adresy
  \item
    Destination - změna cílového portu nebo adresy
  \item
    Maškaráda
  \item
    1:1
  \end{itemize}
\end{itemize}

% nenapádá mě, co sem přidat
% \begin{verbatim}
% TODO
% \end{verbatim}

\#\#~Systém DNS - ``Domain Name System'' - primárně určen pro překlad:
jméno \textless-\textgreater{} adresa - několik typů záznamů: -
\textbf{A} - 32bit IP adresa - \textbf{AAAA} - 128bit IP adresa -
\textbf{MX} - preference a jméno mail serveru - \textbf{TXT} - textový
řetězec - komponenty DNS: - jmenný prostor a zdrojové záznamy - stromová
struktura - jmenné servery - vytváří jmennou databázi, odpovídají na
dotazy - resolvery - komunikace - port 53 UDP (do 512B) i TCP (může i
\textgreater{} 512B) - pokud server nezná odpověď: - rekurzivní chování
- sám najde odpověď a odpoví - nerekurzivní chování - odpoví adresu DNS
serveru kde se má klient ptát - klient může požadovat rekurzivní
chování, server ale může odmítnout

\hypertarget{typy-serverux16f}{%
\subsection{Typy serverů}\label{typy-serverux16f}}

\begin{itemize}
\tightlist
\item
  primární - udržují data o zóně, je autoritativní
\item
  sekundární - kopírují data z primárního serveru, je autoritativní
\item
  caching only - není autoritativní pro žádnou zónu
\item
  root - udržuje záznamy root domény
\item
  forwarding - předává rekurzivní dotaz (odlehčení linky), může sám
  resolvovat
\end{itemize}



\end{document}
