\documentclass{szzclass}
\usepackage{hyperref}


\subject{BEZ}
\code{BI-SPOL-8}
\topic{Infrastruktura veřejného klíče, distribuce klíčů, digitální podpis. Certifikáty, certifikační autority. Kryptograficky bezpečné generátory náhodných čísel.}

\providecommand{\tightlist}{%
  \setlength{\itemsep}{0pt}\setlength{\parskip}{0pt}}

\begin{document}

\tableofcontents
\newpage

\section{Infrastruktura veřejného klíče}
Jedná o specifikaci technických a organizačních opatření pro vydávání, správu, používání
a odvolávání klíčů a certifikátů.

\section{Distribuce klíčů}
Při distribuci veřejného klíče hrozí podvrhnutí prostřednictvím útoku man-in-the-middle.
Jedná se o útok, kde komunikace probíhá přes prostředníka, který zaměňuje předávané klíče za své,
a tím může číst obsah jednotlivých zpráv.

Z tohoto důvodu vzniklo několik způsobů, které tuto situaci řeší.

\section{Techniky distribuce VK}
\subsection{Zveřejnění VK}
\begin{itemize}
  \item VK je zaslán individuálně nebo hromadně v rámci skupiny
  \item vystaví se na internet, dá se do emailu, atd\dots
  \item je to rychlé a jednoduché, ale není odolné proti podvržení
\end{itemize}
\subsection{Veřejně dostupný adresář}
\begin{itemize}
  \item vyšší stupeň bezpečnosti
  \item distribuci zabezpečuje důvěryhodná autorita, která odpovídá za obsah a je správcem adresáře
  \item bezpečná registrace (osobně a nebo přes zabezpečenou komunikaci)
  \item položky jsou v adresáři ukládány jako dvojice [jméno ; VK]
  \item problém může nastat ve chvíli, kdy se odhalí SK patřící správci
\end{itemize}
\subsection{Autorita pro VK}
\begin{itemize}
  \item autorita vykonává činnost správce adresáře
  \item podmínkou je, že každý účastnk zná VK autority
  \item každý účastník musí komunikovat s autoritou
\end{itemize}
\subsection{Certifikace veřejného klíče}
\begin{itemize}
  \item jedná se o distribuci VK, bez kontaktu s třetím důvěryhodným subjektem
  \item vyžaduje se certifikát a certifikační autorita
\end{itemize}
\section{Certifikát}
Jedná se o strukturu, která obsahuje:
\begin{itemize}
  \item VK držitele certifikátu
  \item ID držitele certifikátu
  \item T doba platnosti certifikátu
\end{itemize}
Tato struktura je podepsána soukromým klíčem certifikační autority. Každý účastník může verifikovat obsah certifikátu pomocí veřejného
klíče certifikační autority.
\subsection{Řetězec certifikátů}
\begin{itemize}
  \item posloupnost certifikátů uživatele až ke kořenovému CA
  \item uživatel nemusí věřit CA, stačí pouze ověřit jeden, ne nutně kořenový, certifikát
  \item certifikát je platný $\Leftrightarrow$ platné všechny certifikáty v řetězci certifikátu
  \item pokud existuje více CA pro různé okruhy lidí, vznikají oddělené stromy certifikátů
  \item v případě existence více stromů certifikátů, pomocí křížové certifikace, jednotlivé CA si navzájem podepíší certifikáty
\end{itemize}
\section{Digitální podpis}
Digitální podpis je obvykle formou asymetrického kryptografického schématu. VS slouží k podepsání a VK k ověření.
\newline Musí splňovat následující vlastnosti:
\begin{itemize}
  \item nezfalšovatelnost - podpis se nedá napodobit jiným subjektem než podepisujícím
  \item ověřitelnost - příjemce dokumentu musí být schopen ověřit, že podpis je platný
  \item integrita - podepsaná zpráve se nedá změnit, aniž by se zneplatnil podpis
  \item nepopíratelnost - podepisující nesmí mít později možnost popřít, že dokument podepsal
\end{itemize}

Digitální podpisi se dělí na:
\begin{itemize}
  \item přímé - předají si podpis dvě strany mezi sebou (problém s popíratelností)
  \item verifikované - využívá důvěryhodnou třetí stranu, která ověřuje podpisy všech zpráv
\end{itemize}

\section{Kryptograficky bezpečné generátory náhodných čísel}
Náhodné číslo - čislo vygenerované procesem, který má nepředpovídatelný výsledek a jehož průběh
nelze přesně reprodukovat. Tomuto procesu říkáme generátor náhodných čísel.

Od náhodných posloupností očekáváme dobré statistické vlastnosti:
\begin{itemize}
  \item rovnoměrné rozdělení - všechny hodnoty jsou generovány stejnou pravděpodobností
  \item jednotlivé generované hodnoty jsou nezávislé - není mezi nimi žádná korelace
\end{itemize}

\subsection{Pseudonáhodné generátory}
Jedná se o algoritmicky generovaná "náhodná" čísla. Generátory potřebují náhodný a tajný seed.

Kryptograficky bezpečný PRNG musí splňovat:
\begin{itemize}
  \item next-bit test: pokud se zná prvních n bitů náhodné posloupnost, nesmí existovat algoritmus, který
  v polynomiálním čase dokáže předpovědět další bit, s pravděpodobnstí větsí jak 1/2.
  \item state compromise: i když je znám vnitřní stav generátoru, nelze zpětně zrekonstruovat dosavadní vygenerovanou
  posloupnost. Navíc, pokud do generátoru za běhu vstupuje entropie, nemělo by být možné ze znalosti stavu
  předpovědět stav v dalších iteracích.
\end{itemize}

\subsection{Blum-Blum-Shub}
Jedná se o PRNG, který by měl být kryptograficky bezpečný.
\begin{center}
  $x_{n+1} = x^2_{n-1} mod m$
\end{center}
\begin{itemize}
  \item $x_0$ je definováno seedem a musí být větší 1 (jinak by nefungovalo to umocňování)
  \item modul m = q*r, kde q i r jsou prvočísla
  \item pro q i r musí platit, že q/r = 3 (mod 4)
  \item při znalosti x0 lze dopočítat pomocí rovnice jakýkoliv člen, proto musí zůstat utajen
  \item pokud $x_{n+1}$ vyjde sudé, jde na výstup 0, jinak 1
\end{itemize}
\end{document}
