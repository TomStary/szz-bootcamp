\documentclass{../szzclass}
\usepackage{../dependencies/szz-math}
\usepackage[czech]{babel}
\usepackage[margin=3cm]{geometry}

% \subject{LIN}
% \code{BI-SPOL-13}
% \topic{Matice: součin matic, regulární matice, inverzní matice a její výpočet, vlastní čísla matice a jejich výpočet, diagonalizace matice.}

\begin{document}
\section{Součin matic}
Nechť $m,n,p\in \mathbb{N}$, $\mathbb{A}\in T^{m,n}$ a $\mathbb{B}\in T^{n,p}$. Součinem těchto matic je matice $\mathbb{D}=\mathbb{A}\mathbb{B}$, pro jejíž prvky platí:
$$
\delta_{ij}=\sum_{k=1}^{n}{a_{ik} b_{kj}}
$$

\section{Regulární a inverzní matice}
\begin{definition}[Regulární a inverzní matice]
Nechť $\mathbb{A}\in T^{n,n}$. Pokud existuje $\mathbb{A}\in T^{n,n}$ tak, že 
$$
\mathbb{A}\mathbb{B}=\mathbb{B}\mathbb{A}=\mathbb{E}
$$
Potom nazveme matici $\mathbb{A}$ \textbf{regulární} a matici $\mathbb{B}$ \textbf{inverzní}. Inverzní matici značíme $\mathbb{B}=\mathbb{A}^{-1}$.
\end{definition}

\begin{theorem}
Buď $\mathbb{A}\in T^{n,n}$. Následující tvrzení jsou ekvivalentní:
\begin{itemize}
\item $\mathbb{A}$ je regulární.
\item Soubor řádků matice $\mathbb{A}$ je LN.
\item $h(\mathbb{A}) = n$.
\item $\mathbb{A} \sim \mathbb{E}$.
\end{itemize}
\end{theorem}

\subsection{Výpočet inverzní matice}
\textit{\uv{Inverzní matici regulární matice $\mathbb{A}\in T^{n,n}$ jednoduše spočítám tak, že si část za čárou doplnim o jednotkovou matici $\mathbb{E}$. Pak na to pustim GEM, abych z levý strany dostal $\mathbb{E}$. Na pravý straně mi vyjde $\mathbb{A}^{-1}$.}}
\begin{equation}
\begin{pmatrix}[c]
\mathbb{A}
\end{pmatrix} \rightarrow
\begin{pmatrix}[c|c]
\mathbb{A} & \mathbb{E}
\end{pmatrix} \sim
\begin{pmatrix}[c|c]
\mathbb{E} & \mathbb{A}^{-1}
\end{pmatrix}
\end{equation}

Tohle funguje, protože kroky GEMu se dají \uv{reprezentovat} maticí:
\begin{equation}
\begin{pmatrix}[c|c]
\mathbb{A} & \mathbb{E}
\end{pmatrix} \sim
\begin{pmatrix}[c|c]
\mathbb{P}\mathbb{A} & \mathbb{P}\mathbb{E}
\end{pmatrix} =
\begin{pmatrix}[c|c]
\mathbb{P}\mathbb{A} & \mathbb{P}
\end{pmatrix}
\end{equation}

\section{Vlastní čísla}
\subsection{Výpočet vlastních čísel}
\section{Diagonalizace matice}

\end{document}
