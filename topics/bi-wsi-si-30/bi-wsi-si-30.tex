\documentclass{szzclass}
\usepackage{hyperref}
\usepackage{longtable}
\usepackage{booktabs}

\subject{SI2}
\code{BI-WSI-SI-30}
\topic{Komponenty obchodní logiky EJB. Typy JavaBeans z pohledu uchování stavu (session). Životní cyklus nestavové a stavové EJB. Mechanismus injektování EJB. Popis rozhraní @Local a @Remote.}
\providecommand{\tightlist}{%
  \setlength{\itemsep}{0pt}\setlength{\parskip}{0pt}}

\begin{document}
\tableofcontents
\newpage

\section{EJB}
Enterprise Java Beans jsou serverové komponenty umožňující modulární tvorbu podnikatelských aplikací.
\newline
Specifikace EJB je součástí množiny API definující Java Enterprise Edition (Java EE).
\newline
Cílem EJB je oddělit business logiku aplikace od prezentační a persistentní vrstvy. Poskytuje:
\begin{itemize}
    \item znovupoužitelnost kódu
    \item oddělení logiky aplikace
    \item transakční zpracování
    \item bezpečnost
    \item intergrace s ostatními technologiemi
    \item jednodušší testování
\end{itemize}
Rozdělují se na:
\begin{itemize}
    \item Session Beans
    \begin{itemize}
        \item statefull
        \item stateless
        \item singleton
    \end{itemize}
    \item message driven bean
\end{itemize}
\subsection{EJB kontejner}
Dedikovaný virtuální prostor v aplikačním serveru, kam se nasazují EJB komponenty
\begin{itemize}
    \item komunikace se vzdáleným klientem - zjednodušuje komunikaci mezi klientem a aplikací
    \item dependency injection - zajišťuje naplnění deklarovaných promenných (datových atributů) např. další EJB, datové zdroje (sql spojení)\dots
    \item pooling - vytváření poolu instancí pro bezstavové beany a message-driven beany
    \item řízení životního cyklu - stará se o vytváření, inicializaci a destrukci instancí
\end{itemize}
\subsection{Stateless (bezstavové)}
\begin{itemize}
    \item neuchovávají stav relevatní pro klienta mezi obsluhou jeho jednotlivých požadavků
    \item pouze na žádost něco udělá s daty, které dostane
    \item recyklují se - po vykonání práce může jít do poolu
    \item klientu vždy na serveru přidělena samostatná instance
    \item thread safe
    \item lze použít i datové atributy instance beanu (není zaručeno, že se jejich atribyty při zavolání nezmění)
    \item obsahuje class notaci @Stateless
    \item může obsahovat atributy, které jsou injektované (JDBC spojení)
    \item po kompletním vytvoření se zavolají metody s notací @PostConstruct (lze v nich například zkontrolovat spojení)
    \item když je bean ve stavu method-ready-pool může obsluhovat klientské požadavky
    \item před zníčením se volají metody s anotací @PreDestroy (lze například ukončit DB spojení)
    \item pokud bean vyvolala výjimku = žádné @PreDestroy = okamžité odstranění beany z paměti
\end{itemize}
\subsection{Stateful (stavová)}
\begin{itemize}
    \item uchovávají svůj stav mezi voláním metod (obsluha požadavků klienta) v rámci jendé seassion
    \item anotace třídy je @Statefull
    \item pro každého klienta se tvoří nová instance, při dalším volání metod se použije vždy stejný objekt obsahující data z předchozé interakce klienta
    \item v rámci životního cyklu může dojít k pasivaci = serializaci beanu (odstranění z paměti a uložení na disk) a následně obnovení (děje se tak, při nedostatku paměti)
    \item anotace @PostConstruct a @PreDestroy se chovají jako u stateless
    \item navíc stavy spojené s pasivací - @PrePasivate a @PostActivate
    \item bean může existovat po určitě dlouhou dobu - nastavuje se timeout
\end{itemize}
\subsection{Singleton}
\begin{itemize}
    \item tento EJB má anotaci @Singleton
    \item globálně sdílený stav
    \item bean je synchronyzována - buď si řeší synchronizaci sama nevo pomocí anotace @Lock
\end{itemize}
\subsection{Message driven}
\begin{itemize}
    \item reagují na zprávu
    \item pracují asynchronně - nevrací tedy odpověď
\end{itemize}
\section{Injection EJB}
Do EJB se dá injectovat další EJB nebo i jiné třídy. K naplnění proměnných dochází při vytváření beany - vše zajišťuje EJB kontejner.
Používá JNDI pro vyhledávání
\begin{itemize}
    \item @EJB - anotace pro injectování jiné beany
    \item @Resource - anotace pro injectování zdrojů dat nebo kontextu
\end{itemize}
\section{Rozhraní}
Definuje co musí daná bean implementovat, každá musí být dostupná lokálně nebo vzdáleně (nebo obojí)
\subsection{Lokální}
\begin{itemize}
    \item implicitní chování, pokud není jinak specifikováno
    \item lokální dostupnosti se většinou myslí, že je dostupná pro ostatní komponenty běžící na stejné JVM
    \item anotace @loacl
\end{itemize}
\subsection{Remote}
\begin{itemize}
    \item pro volání vzdálené JVM
    \item například EJB běží na aplikačním serveru a komunikuje s PC uživatele
    \item získání instance pomocí JNDI
    \item anotace @Remote
\end{itemize}
\end{document}