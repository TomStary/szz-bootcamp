\documentclass{szzclass}
\usepackage{dependencies/szz-code}
\usepackage[czech]{babel}

\subject{DBS}
\code{BI-SPOL-9}
\topic{Relační databáze, dotazování v relační algebře, základní koncepce jazyka SQL (SELECT, DDL, DML, DCL, TCL) , vyjádření integritních omezení v DDL.}

\begin{document}
\section{Relační databáze}
\section{Dotazování v relační algebře}
\begin{itemize}
  \item Relační algebra je pouze dotazovací formalismus. (Pokrývá jenom \mintinline{sql}{SELECT} v SQL.)
  \item 
\end{itemize}
\section{Základní koncepce jazyka SQL (\mintinline{sql}{SELECT}, DDL, DML, DCL, TCL)}
\begin{description}
  \item[DDL] Definiční jazyk -- např. manipulace s tabulkama, integritní omezení
  \item[DML] Manipulační jazyk -- např. \mintinline{sql}{SELECT}, \mintinline{sql}{INSERT}, \mintinline{sql}{UPDATE} apod.
  \item[DCL] Jazyk na přístupy
  \item[TCL] Jazyk pro řízení transakcí
\end{description}

\begin{minted}{sql}
SELECT sloupce
FROM tabulky
[WHERE podmínky]
[ORDER BY řazení]
\end{minted}

\section{Vyjádření integritních omezení v DDL}

\end{document}
