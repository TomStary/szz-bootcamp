\documentclass{szzclass}

\title{Lambda kalkul: definice pojmů, operací, reprezentace čísel.}
\author{
    Tomáš Starý
    \and
    Daniel Hampl
}

\begin{document}
\maketitle

\tableofcontents
\newpage

\section{Definice}

Lambda ($\lambda$) kalkul je formální systém používaný v teoretické informatice a matematice pro studium funkcí a rekurze, lambda kalkul je Turing-complete.
Funkce zapsané v lambda kalkulu lze poté vyhodnotit pomocí substituce.

\subsection{Syntaxe lambda kalkulu}

U lambda kalkulu se používá prefixový zápis, tedy operátory se píší před operandy, například (+(* 5 3)(* 5 3)).
Vyhodnocení poté probíhá zleva doprava a to následovně:
\begin{enumerate}
    \item (+ 30 (* 5 3))
    \item (+ 30 30)
    \item 60
\end{enumerate}

Zápis funkce s proměnnou se poté zapisuje takto: $(\lambda x. + x\ 1)$.

\subsection{Beta redukce}

Vyhodnocování funkcí v lambda kalkulu se dělá pomocí beta ($\beta$) redukce.
Beta redukce provedená na příkladu zápisu funkce:
\begin{enumerate}
    \item $(\lambda x. + x\ 1)$
    \item $(\lambda x. + x\ 1)2$
          2 za závorkou je argumentem fuknce.
    \item $(+ 2 1)$
    \item 3
\end{enumerate}

Další platné zápisy: $(\lambda x. + x\ x) 2 => (+\ 2\ 2)$.

\subsection{Volné a vázáné proměnné}

V lambda kalkulu rozlišuje dva typy proměnných a to vázané a volné. Vázané proměnné jsou takové proměnné, které jsou zároveň argumentem
dané funkce. Lambda kalkul má lokální rozsah platnostnosti (scope).
Například pro fuknci $(\lambda x . + x\ y)$ je vázanou proměnnou x a y je v tomto případě volná proměnná.

\newpage

\subsection{Alfa redukce}

Alfa $(\alpha)$ redukce odstraňuje přetížení identifikátorů pomocí přejmenování argumetu a jeho vázaných proměnných.
Ukázka na $(\lambda x . (\lambda x . +\ (-\ x\ 1))\ x\ 3)\ 9$.
\begin{enumerate}
    \item $(\lambda y . +\ (-\ y\ 1)\ 9\ 3)$
    \item $(+\ (-\ 9 1)\ 3)$
    \item $(+\ 8\ 3)$
\end{enumerate}


\subsection{Eta redukce}
Eta $(\eta)$ redukce (optimalizace): $(\lambda x. fx) = f$ pokud se x nevyskytuje nikde volně v $f$.

\begin{enumerate}
    \item $(\lambda x.(\lambda y.yy)x)$
    \item $(\lambda y. yy)$ (po $\eta$ redukci)
\end{enumerate}


\subsection{Pořadí vyhodnocování}

Rozlišujeme 2 možná pořadí vyhodnocování, normální (lazy) a aplikativní (strict), přičemž platí, že pokud lze výraz
vyhodnotit více způsoby a výsledný výraz je v normálním tvaru, pak všechny způsoby produkují stejný výsledek.

Normální vyhodnocování probíhá tak, že nejdříve nalezneme nejlevější $\lambda$ a její argumenty a proveď na nich substituci.
Pro aplikativní vyhodnocování se postupuje tak, že opět nalezneme nejlevější $\lambda$ a její aplikuj její argumenty, pak proveď substituci.
U aplikativního vyhodnocování může dojít k zacyklení.

Normálním tvarem poté pokládáme takový tvar lambda výrazu, na kterém již nelze provádět beta redukci.

\subsection{Zápis čísel v lambda kalkulu}

Churchova čísla je způsob jak reprezentovat přirozená čísla v $\lambda$-kalkulu. Každé číslo je reprezentováno jako funkce s dvěma parametry, první parametr se  $n$-krát opakuje, druhý je jakousi "zarážkou".

\begin{itemize}
    \item $0 = (\lambda s.(\lambda z.z))$
    \item $1 = (\lambda s.(\lambda z. (s z)))$
    \item $2 = (\lambda s.(\lambda z. (s (s z))))$
    \item $3 = (\lambda s.(\lambda z. (s (s (s z)))))$
\end{itemize}

Případně ve zkráceném zápisu: $\lambda sz.z = 0 | \lambda sz. s(z) = 1$

Vybrané operace s čísly pak zavádíme následovně:
\begin{itemize}
    \item $+~x~1 =~$succ$~= (\lambda x. \lambda s. \lambda z. s (x s z))$
    \item $-~x~1 =~$pred$~= (\lambda x. \lambda s. \lambda z. x (\lambda f. \lambda g.g (f s)) (\lambda g.z) (\lambda m.m))$
    \item $+~x~y =~$add$~= (\lambda x. \lambda y. \lambda s. \lambda z. x s (y s z))$
    \item $*~x~y =~$mult$~= (\lambda x. \lambda y. \lambda z. x (y z)) nebo * x y = (\lambda x. \lambda y. \lambda s. \lambda z. x (y s) z)$
\end{itemize}

\end{document}
