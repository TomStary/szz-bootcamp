\documentclass{szzclass}
\usepackage{hyperref}
\usepackage{longtable}
\usepackage{booktabs}

\subject{SI2}
\code{BI-WSI-SI-25}
\topic{Konfigurační řízení: řízení verzí, řízení změn, správa prostředí, continuous integration, řízení dodávek, vzájemné souvislosti.}

\providecommand{\tightlist}{%
  \setlength{\itemsep}{0pt}\setlength{\parskip}{0pt}}

\begin{document}
\tableofcontents
\newpage

\section{Konfigurační řízení}
Jedna z nejdůležitejších podpůrných činností v rámci SW vývoje. Cílem je zajistit řád a pořádek v konfiguraci SW produktu.
\subsection{Základní aktivity}
\begin{itemize}
  \item řízení změn
  \item verzování
  \item release managment
\end{itemize}
\subsection{Cíle}
\begin{itemize}
  \item evidence všech částí SW produktu
  \item zajistit, že provádění změn SW produktu samotný produkt nepoškodí
  \item získat přehled o stavu konfigurace SW produktu
\end{itemize}
\subsection{Kontrola verzí}
\begin{itemize}
  \item evidence SW položek
  \item práce na více verzích současně
  \item obnovení konkrétní verze
  \item GIT, SVN
  \item branch (vedlejší větev/větve), tag (read-only copy/záložka mezi verzemi), trunk (hlavní větev) - v pojetí SVN
  \item konfigurační jednotka - libovolný soubor, který se verzuje
\end{itemize}
\subsection{Řízení změn}
\begin{itemize}
  \item proces, který změnu zakomponuje do projektu se všemi náležitostmi (včetně například testování)
  \item eviduje se typ změny a vazba na specifikaci
  \item Ticketovací systém - youtrack/redmine/github/bugzilla\dots
\end{itemize}
\section{Správa prostředí}
Software poběží u zákazníka, ale musí se někde vyvíjet a testovat, tzn. existuje více prostředí, kde program poběží.
\subsection{U dodavatele}
\begin{itemize}
  \item vývoj
  \begin{itemize}
    \item lokální vývoj, programátorovi by se mělo programovat pohodlně
    \item povinná sada testů před commitem - měly by být krátké a rychlé 
  \end{itemize}
  \item integrační
  \begin{itemize}
    \item continuous integration - provedení smoke testů a daily build
    \item po smoke testech (pokud projdou) se provedé rozsáhlé testování automatizovanými testy
  \end{itemize}
  \item testovací
  \begin{itemize}
    \item funkční/nefunkční testy
    \item manuální testy
  \end{itemize}
\end{itemize}
\subsection{U zákazníka}
\begin{itemize}
  \item akceptační - testuje prostředí u zákazníka
  \item produkční - testuje se prostředí, na kterém běží výsledná aplikace
\end{itemize}
\section{Řízení dodávek}
Jak často se budou zákazníkovi poskytovat nové buildy záleží na zvoleném modelu SDCL modelu (agile, waterfall\dots).
Většinou cílem je automatizace procesu dodávek.
\subsection{Proces}
\begin{itemize}
  \item vyrobit dodávku
  \item nainstalovat dodávku
  \item otestovat dodávku
  \item připravit dodávku pro instalaci zákazníkem
\end{itemize}
\section{Continuous integration}
\begin{itemize}
  \item souhrn různých vývojářských nástrojů a metod k urychlení vývoje SW
  \item integrace kodu do celkového systému na každodenní bázi (většinou pomocí build serveru)
  \item při mnoha commitech se sloučí do skupin, pro menší zátež (někdy i rychlejší deploy při mnoha testech)
  \item jako první se provedou smoke testy $\rightarrow$ pokud projdou $\rightarrow$ ostatní testy $\rightarrow$ pokud projdou $\rightarrow$ build a deploy
\end{itemize}
\end{document}