\documentclass{szzclass}
\usepackage{dependencies/szz-math}
\usepackage{bbm}
\usepackage[czech]{babel}
\usepackage[margin=3cm]{geometry}

\subject{LIN}
\code{BI-SPOL-12}
\topic{Soustavy lineárních rovnic: Frobeniova věta a související pojmy, vlastnosti a popis množiny řešení, Gaussova eliminační metoda.}

\begin{document}
\section{Frobeniova věta}
\begin{definition}
Nechť $\mathbb{A}\in T^{m,n}$. \textbf{Hodností matice} $\mathbb{A}$ nazýváme dimenzi lineárního obalu řádků matice $\mathbb{A}$ (jako vektorů z $T^{n,n}$) a značíme ji $h(\mathbb{A})$:
\begin{equation}
  h(\mathbb{A})=dim\langle \mathbb{A}_{1:},\dots,\mathbb{A}_{m:},\rangle.
\end{equation}
\end{definition}

\begin{theorem}[Frobeniova věta]
Nechť $\mathbb{A}\in T^{m,n}$.
\begin{enumerate}
\item Soustava $m$ lineárních rovnich o $n$ neznámých $\mathbb{A} \mathbbm{x}=\mathbbm{b}$ je řešitelná právě tehdy, když
$$
h(\mathbb{A})=h(\mathbb{A} | \mathbbm{b})
$$
\item Je-li $h(\mathbb{A})=h$, pak množina řešení $\mathbb{A} \mathbbm{x}=\theta$ je podprostor dimenze $n-h$, tedy existuje LN soubor vektorů $(z_1,\dots,z_{n-h})$ v $T^{n,1}$, takový, že
$$ S_0=\begin{cases}
       \{\theta\}, & \text{pokud $n=h$,} \\
       \langle z_1,\dots,z_{n-h}\rangle, & \text{pokud $h<n$.}
       \end{cases}
$$
Je-li navíc $h(\mathbb{A}|\mathbbm{b})=h$, potom platí:
$$
S=\widetilde{x} + S_0,
$$ kde $\widetilde{x}$ je \textbf{partikulární řešení} $\mathbb{A} \widetilde{\mathbbm{x}}=\mathbbm{b}$.
\end{enumerate}
\end{theorem}

\section{Vlastnosti a popis množiny řešení}
Soustava lineárních rovnic (zapsaných v HST matice $(\mathbb{A}\mathbbm{x}|\mathbbm{b})$) může mít:
\begin{itemize}
\item Žádné řešení -- poslední sloupec je hlavní
\item Jedno řešení -- všechny sloupce, kromě posledního jsou hlavní
\item Více řešení -- sloupce jsou hlavní nebo vedlejší (poslední sloupec nesmí být hlavní)
Pro popis množiny řešení se nalezne obal LN souboru možných řešení. \emph{viz. Frobeniova věta}
\end{itemize}

\section{Soustavy lineárních rovnic}
\begin{itemize}
\item Přepíšu soustavu do matice $(\mathbb{A}|\mathbbm{b})$.
\item Převedu do HST pomocí GEM.
\item Poslední sloupec hlavní $\rightarrow$ nemá řešení.
\item Jinak:
  \begin{itemize}
  \item Najdu volné a vázané proměnné (odpovídá vedlejším a hlavním sloupcům).
  \item Pro partikulární řešení zvolim volné proměnné libovolně a dopočítám vázané proměnné.
  \item Pro $S_0$ zvolim libovolnou bázi (třeba standardní).
  \item Pro každý bazický vektor dopočítám z homogenní rovnice vázané proměnné a dostanu bázi $S_0$.
  \item Řešením je $S=\widetilde{\mathbbm{x}}+S_0$, kde $S_0$ je lineární obal báze.
  \end{itemize}
\end{itemize}

\section{Gaussova eliminační metoda}
Cílem GEM je převést matici do horního stupňovitého tvaru, pomocí úprav $(G1)$, $(G2)$ a $(G3)$.

\begin{itemize}
\item[$(G1)$] Prohození dvou řádků.
\item[$(G2)$] Vynásobení jednoho řádku nenulovým číslem.
\item[$(G3)$] Přičtení jednoho řádku k jinému.
\end{itemize}

\end{document}
