\documentclass{szzclass}
\usepackage{dependencies/szz-math}
\usepackage[czech]{babel}
% \usepackage{bbm}

\newcommand{\mathbbm}[1]{\boldsymbol#1}

% Augmented pmatrix - https://latex.org/forum/viewtopic.php?f=19&t=22393
\makeatletter
\renewcommand*\env@matrix[1][*\c@MaxMatrixCols c]{%
   \hskip -\arraycolsep
   \let\@ifnextchar\new@ifnextchar
   \array{#1}}
\makeatother

\subject{LIN}
\code{BI-SPOL-12}
\topic{Soustavy lineárních rovnic: Frobeniova věta a související pojmy, vlastnosti a popis množiny řešení, Gaussova eliminační metoda.}

\begin{document}

\section{Frobeniova věta}
\begin{definition}
Nechť $\mathbb{A}\in T^{m,n}$. \textbf{Hodností matice} $\mathbb{A}$ nazýváme dimenzi lineárního obalu souboru řádků matice $\mathbb{A}$ (jako vektorů z $T^{1,n}$) a značíme ji $h(\mathbb{A})$:
\begin{equation}
  h(\mathbb{A})=dim\langle \mathbb{A}_{1:},\dots,\mathbb{A}_{m:}\rangle.
\end{equation}
\end{definition}

\begin{theorem}[Frobeniova věta]
Nechť $\mathbb{A}\in T^{m,n}$.
\begin{enumerate}
\item Soustava $m$ lineárních rovnich o $n$ neznámých $\mathbb{A} \mathbbm{x}=\mathbbm{b}$ je řešitelná, tj. $S\neq\theta$, právě tehdy, když
$$
h(\mathbb{A})=h(\mathbb{A} | \mathbbm{b}).
$$
\item Je-li $h(\mathbb{A})=h$, pak množina řešení $\mathbb{A} \mathbbm{x}=\theta$ je podprostor dimenze $n-h$, tedy existuje LN soubor vektorů $(\mathbbm{z}_1,\dots,\mathbbm{z}_{n-h})$ v $T^{n,1}$ takový, že
$$ S_0=\begin{cases}
       \{\theta\}, & \text{pokud $n=h$,} \\
       \langle \mathbbm{z}_1,\dots,\mathbbm{z}_{n-h}\rangle, & \text{pokud $h<n$.}
       \end{cases}
$$
Je-li navíc $h(\mathbb{A}|\mathbbm{b})=h$, potom platí:
$$
S=\widetilde{\mathbbm{x}} + S_0,
$$ kde $\widetilde{\mathbbm{x}}$ je \textbf{partikulární řešení} $\mathbb{A} \widetilde{\mathbbm{x}}=\mathbbm{b}$.
\end{enumerate}
\end{theorem}

\section{Vlastnosti a popis množiny řešení}
Soustava lineárních rovnic (zapsaných v HST matice $\begin{pmatrix}[c|c]\mathbb{A}\mathbbm{x}&\mathbbm{b}\end{pmatrix}$) může mít:
\begin{itemize}
\item Žádné řešení -- poslední sloupec je hlavní
\item Jedno řešení -- všechny sloupce, kromě posledního jsou hlavní
\item Více řešení -- sloupce jsou hlavní nebo vedlejší (poslední sloupec nesmí být hlavní)
Pro popis množiny řešení se nalezne obal LN souboru možných řešení. \emph{viz. Frobeniova věta}
\end{itemize}

\section{Soustavy lineárních rovnic}
\begin{definition}
Nechť $n,m\in\mathbb{N}$ a pro všechna $i\in\{1,\dots, m\}$ a
$j\in\{1, \dots, n\}$ platí, že $a_{ij} \in\mathbb{R}$ a $b_i \in\mathbb{R}$. Systém rovnic

% https://en.wikipedia.org/wiki/System_of_linear_equations
\begin{alignat}{7}
a_{11} x_1 &&\; + \;&& a_{12} x_2 &&\; + \cdots + \;&& a_{1n} x_n &&\; = \;&&& b_1 \\ \notag
a_{21} x_1 &&\; + \;&& a_{22} x_2 &&\; + \cdots + \;&& a_{2n} x_n &&\; = \;&&& b_2 \\ \notag
\vdots\;\;\; &&     && \vdots\;\;\; &&  \ddots\;\;\;&& \vdots\;\;\; &&     &&& \,\vdots \\ \notag
a_{m1} x_1 &&\; + \;&& a_{m2} x_2 &&\; + \cdots + \;&& a_{mn} x_n &&\; = \;&&& b_m \\ \notag
\end{alignat}

nazýváme \textbf{soustavou} $m$ \textbf{lineárních rovnic o} $n$ \textbf{neznámých} $x_1,\dots, x_n \in\mathbb{R}$.\\
Číslu $aij$ říkáme $j$\textbf{tý koeficient} $i$\textbf{té rovnice}.

Množinu všech uspořádaných $n$tic $(x_1, x_2,\dots, x_n) \in\mathbb{R}^n$, pro které po dosazení do (2) je splněno všech $m$ rovnic, nazýváme \textbf{množinou řešení soustavy} a značíme ji $S$.
Platí-li $b_1 = b_2 = \dots = b_m = 0$, říkáme, že soustava (2) je \textbf{homogenní}.
Není-li soustava homogenní, je \textbf{nehomogenní}.
\end{definition}

\subsection{Horní stupňovitý tvar}
\begin{definition}
O matici $\mathbb{D}\in R^{m,n}$ řekneme, že je v \textbf{horním stupňovitém tvaru}, jestliže všechny řádky jsou nulové, nebo existuje $k\in\hat{m}$ tak, že řádky $1$ až $k$ matice $\mathbb{D}$ jsou nenulové a řádky $k + 1$ až $m$ jsou nulové a jestliže platí následující:

Označme pro každé $i\in\hat{k}$ index nejlevějšího nenulového prvku v $i$tém řádku jako
$j_i$, tj.

$$
j_i=\min{\{\ell\in\hat{n}|\mathbb{D}\neq 0\}}.
$$
Potom platí $1 \leq j_1 < j_2 < \cdots < j_k$.

Je-li matice v horním stupňovitém tvaru, potom sloupcům s indexy $j_1, j_2, \dots , j_k$
říkáme \textbf{hlavní sloupce}, ostatním říkáme \textbf{vedlejší sloupce}.

O soustavě $\mathbb{A}\mathbbm{x} = \mathbbm{b}$ řekneme, že je v horním stupňovitém tvaru, pokud matice této soustavy $(\mathbb{A}|\mathbbm{b})$ je v horním stupňovitém tvaru.
\end{definition}

\subsection{Postup pro řešení SLR}
\begin{itemize}
\item Převedu SLR do matice $(\mathbb{A}|\mathbbm{b})$.
\item Převedu do HST pomocí GEM.
\item Poslední sloupec hlavní $\rightarrow$ nemá řešení.
\item Jinak:
  \begin{itemize}
  \item Najdu volné a vázané proměnné (odpovídá vedlejším a hlavním sloupcům).
  \item Pro partikulární řešení zvolim volné proměnné libovolně a dopočítám vázané proměnné.
  \item Pro $S_0$ zvolim libovolnou bázi (třeba standardní).
  \item Pro každý bazický vektor dopočítám z homogenní rovnice vázané proměnné a dostanu bázi $S_0$.
  \item Řešením je $S=\widetilde{\mathbbm{x}}+S_0$, kde $S_0$ je lineární obal báze.
  \end{itemize}
\end{itemize}

\section{Gaussova eliminační metoda}
Cílem GEM je převést matici do horního stupňovitého tvaru, pomocí úprav $(G1)$, $(G2)$ a $(G3)$.

\begin{itemize}
\item[$(G1)$] Prohození dvou řádků.
\item[$(G2)$] Vynásobení jednoho řádku nenulovým číslem.
\item[$(G3)$] Přičtení jednoho řádku k jinému.
\end{itemize}

\end{document}
