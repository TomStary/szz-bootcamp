\documentclass{szzclass}
\usepackage[czech]{babel}

\title{Výroková logika: syntax a sémantika výrokových formulí, pravdivostní ohodnocení, logický důsledek, ekvivalence a jejich zjišťování. Universální systém logických spojek, disjunktivní a konjunktivní normální tvary, úplné a minimální tvary.}
\author{Jakub Rathouský}

\begin{document}
\maketitle
\tableofcontents
\newpage

\section{Výroková logika: syntax a sémantika výrokových formulí}
\subsection{Prvotní výrok a formule}
\textbf{Prvotní výrok} je jednoduchá oznamovací věta, u které má smysl se ptát, zda je či není pravdivá.
Prvotní výrok se označuje velkým tiskacím písmenem A, B\dots, kterým se říká \textbf{prvotní formule}.
\subsection{Negace}
($\neg$) - negace formule je pravdivá, právě když je formule nepravdivá.
\subsection{Konjunkce}
($\wedge$) - konjunkce dvou formulí je pravdivá tehdy, když jsou obě formule pravdivé.
\subsection{Disjunkce}
($\vee$) - disjunkce dvou formulí je pravdivá, právě když alespoň jedna z nich je pravdivá.
\subsection{Implikace}
($\Rightarrow$)
\begin{itemize}
	\item implikace je nepravdivá tehdy, když předpoklad je pravdivý a závěr nepravdivý
	\item implikace je pravdivá tehdy, když neplatí předpoklad nebo platí závěr
\end{itemize}
\subsection{Formule výrokové logiky}
\subsubsection{Jazyk výrokové logiky}
\begin{itemize}
	\item symboly pro prvotní formule A, B, \dots
	\item logické spojky $\neg, \wedge, \vee, \Rightarrow, \Leftarrow$
	\item závorky ()
\end{itemize}
\subsubsection{Formule výrokové logiky}
Je definovaná:
\begin{itemize}
	\item prvotní formule je výroková formule
	\item jsou-li A a B výrokové formule, pak jsou i $ \neg A, (A \wedge B), (A \vee B), (A \Rightarrow B)$ výrokové formule
	\item formule je řetězec symbolů sestavený podle předchozích 2 pravidel v konečně mnoha krocích
\end{itemize}
\subsection{Ekvivalence}
($\Leftrightarrow$) - ekvivalence dvou formulí je pravdivá právě tehdy, když obě mají stejnou pravdivostní hodnotu
\section{Pravdivostní ohodnocení}
\textbf{Pravdivostní ohodnocení} množiny prvotních výroků je funkce \textit{v} z množiny prvotních formulí do množiny \{0,1\}.
\\
\textit{v}: $\{A_1,...,A_n\}\rightarrow \{0,1\}$
\\
Je-li \textit{v}(A) = 1, řekne se, že A je \textbf{pravdivý při ohodnocení} \textit{v}.\\
Je-li \textit{v}(A) = 0, řekne se, že A je \textbf{nepravdivý při ohodnocení} \textit{v}.
\section{Logický důsledek}
Formule B je logickým důsledkem formule A, právě když pro každé ohodnocení v, pro které v(A) = 1,
je i v(B) = 1.
\\
Píše se A $\models$ B.
\subsection{Vztah logického důsledku a logické ekvivalence}
Pro každé dvě formule výrokové logiky A, B platí:
\begin{itemize}
	\item $A \models B \wedge B \models A$ právě, když $A \Leftrightarrow B$ je tautologie
	\item $A \models B$ právě, když $A \Rightarrow B$ je tautologie
	\item $A \models B$ právě, když $A \wedge \neg B$ je kontradikce
\end{itemize}
\section{Ekvivalence a jejich zjišťování}
Formule A a B jsou \textbf{logicky ekvivaletní} právě tehdy, když pro každé ohodnocení v je v(A) = v(B). Píšeme $A \models B \wedge B \models A$

\subsection{Zjišťování}
\begin{itemize}
	\item pomocí porovnání pravdivosti výroků
	\item úpravou formulí, převedením na sebe
\end{itemize}

\section{Universální systém logických spojek}
Množina logických spojek tvoří universální systém, právě když ke každé formuli existuje logicky ekvivaletní formule, která obsahuje pouze
tyto spojky. Například:
\begin{itemize}
	\item \{$\neg, \vee$\}
	\item \{$\neg, \wedge $\}
	\item \{$\neg, \Rightarrow$\}
	\item NAND \{$\uparrow$\} = $\neg(A \wedge B)$
	\item NOR \{$\downarrow$\} = $\neg(A \vee B)$
\end{itemize}
\section{Disjunktivní a konjunktivní normální tvary}
\subsection{DNT - disjunktivní normální tvar}
\begin{itemize}
	\item \textbf{literál} je prvotní formule nebo negace prvotní formule
	\item \textbf{implikant} je literál nebo konjunkce několika literálů
	\item formule je v \textbf{DNT}, jestliže je implikant nebo disjunkce několika implikantů 
\end{itemize}

ukázka:
\begin{itemize}
	\item A, $\neg B$ - literál
	\item $A \wedge B$, $A \wedge \neg B$, $\neg B$ - implikant
	\item A, $(A \wedge \neg B) \vee (A \wedge C)$, $A \vee \neg B$, $A \wedge \neg B$ - DNT
\end{itemize}
\subsection{KNT - konjunktivní normální tvar}
\begin{itemize}
	\item \textbf{literál} je prvotní formule nebo negace prvotní formule
	\item \textbf{klausule} je literál nebo disjunkce několika literálů
	\item formule je v \textbf{KNT}, jestliže je klausulí nebo konjunkce několika klausulí 
\end{itemize}

ukázka:
\begin{itemize}
	\item A, $\neg$B - literál
	\item A$\vee$B, A$\vee$$\neg$B, $\neg$B - klausule
	\item A, $(A \vee \neg B) \wedge (A \vee C)$, $A \vee\neg B$, $A \wedge \neg B$ - KNT
\end{itemize}

\subsection{POZOR}
Některé vybrané KNT jsou i DNT (a obráceně)! Například:
\begin{itemize}
	\item A
	\item A$\wedge$B
	\item A$\vee$B
\end{itemize}
\subsection{Existence DNT a KNT}
Ke každé formuli existuje formule logicky ekvivalentní, která je v DNT, a formule logicky ekvivaletní, která je v KNT.
(Důkaz na slidu 11 přednáška 3 - BI-MLO)
\section{Úplné a minimální tvary}
\begin{itemize}
	\item \textbf{Minterm} formule A je implikant, který obssahuje všechny prvotní formule vyskytující se v A
	\item \textbf{Maxtern} formule A je klausule, která obsahuje všechny prvotní formule, vyskytující se v A
	\item formule je v \textbf{úplném disjunktivním normálním tvaru}, jestliže je disjukcí mintermů.
	\item formule je v \textbf{úplném konjunktivním normálním tvaru}, jestliže je konjunkcí maxtermů.
\end{itemize}
\subsection{Existence úplného DNT a KNT}
Ke každé formuli existuje formule logicky ekvivaletní, která je v úplném DNT, a formule logicky ekvivaletní, která je v úplném DNT.

Úplný KNT i DNT libovolné formule je dán jednoznačně až na pořadí (literálů, mintermů, maxtermů). Pokud má formule \textit{n} prvotních
formulí, pak součet mintermů a maxtermů je $2^n$.
\subsection{Ekvivalence ÚDNT a ÚKNT}
Následující tvrzení jsou ekvivaletní:
\begin{itemize}
	\item $A \models B \wedge B \models A$
	\item ÚDNT obsahují stejné mintermy
	\item ÚKNT obsahují stejné maxtermy
\end{itemize}
\subsection{Logický důsledek a ÚDNT/ÚKNT}
Vezmou se dvě formule A a B, které obsahují stejné prvotní formule. $A_d, A_k, B_d, B_d$ jsou jejich ÚDNT a ÚKNT.
Následující tvrzení jsou ekvivaletní
\begin{itemize}
	\item $A \models B$
	\item Všechny mintermy $A_d$ jsou obsaženy v $B_d$
	\item Všechny maxterny $B_k$ jsou obsaženy v $A_k$
\end{itemize}
\end{document}
