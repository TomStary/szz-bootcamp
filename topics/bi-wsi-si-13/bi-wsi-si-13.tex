\PassOptionsToPackage{table}{xcolor}
\documentclass{szzclass}

\subject{PPA}
\code{BI-WSI-SI-13}
\topic{Rozdělení paměti při implementaci programovacích jazyků: statické části, zásobník, halda. Aktivační záznamy, mechanismus implementace volání funkcí.}

\begin{document}

\section{Rozdělení paměti při implementaci programovacích jazyků}
Běh programu běží v souvislém logickém adresovém prostoru, který je poskytován operačním systémem. Operační systém mapuje logické adresy na fyzické adresy (které nemusí tvořit souvislý celek).

\begin{center}
\def\arraystretch{2}
\begin{tabular}{|c|} \hline
Kód programu \\ \hline
Statické objekty \\ \hline
Halda (Heap) \\ \hline
Volná paměť \\ \hline
Zásobník (Stack) \\ \hline
\end{tabular}
\end{center}

\subsection{Statické objekty}
\begin{itemize}
\item Globální konstanty, globální proměnné, data generované překladačem,...
\item Výhody
  \begin{itemize}
  \item Nepotřebuje správu paměti při běhu programu
  \item Rychlý přístup (překládá se přímo adresou v paměti)
  \item Není nebezpečí out-of-memory
  \end{itemize}
\item Nevýhody
  \begin{itemize}
  \item Velikost a počty musí být známy předem
  \end{itemize}
\end{itemize}

\subsection{Objekty na zásobníku}
\begin{itemize}
\item Každé volání funkce má svůj aktivační záznam (activation record). Podobně každý vnitřní blok.
\item LIFO přístup podporuje volací a návratový mechanismus volání funkcí a procedur.
\item Přirozená podpora volání funkcí a návratu z funkcí. Podpora rekurze.
\item Objekty jsou vloženy na zásobník při vstupu do funkce a jsou ze zásobníku odebrány při návratu.
\item Volání funkce je implementováno volací sekvencí, návrat je implementován návratovou sekvencí.
\end{itemize}

\subsection{Halda}
\begin{itemize}
\item Obsahuje dynamické objekty/proměnné vytvořené během běhu programu, např. pomocí příkazů malloc a free (C) nebo new a delete (C++).
\item V některých programovacích jazycích je halda udržována pomocí garbage collection.
\end{itemize}

\subsection{Aktivační záznam pro procedury a funkce}
\begin{itemize}
\item Hodnotou
  \begin{itemize}
  \item Skutečná hodnota je vypočítána a následně zkopírována.
  \item Vstupní parametr, chová se jako lokální proměnná.
  \end{itemize}
\item Odkazem
  \begin{itemize}
  \item Volající určuje adresu v paměti.
  \item Vstupní/výstupní proměnné.
  \end{itemize}
\end{itemize}

\subsection{Aktivační záznam}
\begin{itemize}
\item S každým voláním funkce se na zásobníku vytvoří aktivační záznam.
\item Jazyky bez vnořených procedur, např. C
\item Lokální statické proměnné jsou uloženy v lokálním aktivačním záznamu na zásobníku.
\item Jazyky s vnořenými procedurami, např. Pascal, také Lisp a funkcionální jazyky obecně, kde funkce typicky vytváří jinou funkci
\item Musí být dále přidán link mezi aktivačními záznamy procedur (funkcí).
\end{itemize}

\subsection{Mechanismus implementace volání funkcí}
\begin{center}
\def\arraystretch{2}
\begin{tabular}{|c|} \hline
Skutečné parametry \\ \hline
Návratové hodnoty \\ \hline
\rowcolor{lightgray} Řídící link (link na starý aktivační záznam) \\ \hline
Přístupový link (link na aktivační záznam vyšší procedury) \\ \hline
\rowcolor{lightgray} Uložené údaje počítače (registry, atd\dots) \\ \hline
Lokální proměnné \\ \hline
Pomocné dočasné proměnné \\ \hline
\end{tabular}
\end{center}

\begin{itemize}
\item Stack pointer SP - Odkaz na vrchol zasobníku
\item Frame Pointer FP - Odkaz na začátek aktualního záznamu
\end{itemize}

\end{document}

% Z přednáškových prezentací:
% Imperativní (procedurální) programy jsou popisem akcí, které mění stav (který je dán obsahem paměti).
% - Základní je přiřazovací příkaz, používají se typicky vedlejší efekty (side effects), iterační cykly.
% - Fortran, Algol, Cobol, Pascal, Modula-2, Ada, C. (OOP rozšíření: C++, C#, ...)
% - Dominující paradigma, které je přirozené pro von Neumannovu architekturu počítače.
% - Odpovídající základní model výpočtu Turingův stroj (viz AAG)
% - Typické fičury:
%    - Klíčová operace: přířazovací příkaz
%    - Vedlejší efekty
%    - Podmíněné skoky, cykly, podmínky
%    - Datové typy, deklarace proměnných, bloky

% Objektově orientované (OO) programování nemusí být vždy imperativní, ale většina OO jazyků je imperativních
% - Spojení dat a algoritmů ve formě tříd, dědičnost, polymorfismus, ...
% - Smalltalk, C++, Modula-3, Java, C#
% - OO funkcionální jazyk: CLOS (Common Lisp Object System) 

% Funkcionální programování je zaměřeno na funkce a jejich vyhodnocování.
% - Nemá přiřazovací příkaz, nemá vedlejší efekty, typické používání rekurze.
% - Má funkce vyšších řádů: parametrem i výsledkem funkce může být funkce. Má typicky uzávěry (closures).
% - LISP, Haskell, R, Clojure, Python
% - Odpovídající základní model výpočtu lambda kalkulus (viz PPA).

% Logické programování je založeno na predikátové logice
% - Vhodné zejména ve specifických aplikačních oblastech: databázové aplikace, automatické dokazování, umělá inteligence, ...
% - Prolog
