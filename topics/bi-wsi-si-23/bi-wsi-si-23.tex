\documentclass{szzclass}
\usepackage{hyperref}
\usepackage{longtable}
\usepackage{booktabs}

\subject{SI2}
\code{BI-WSI-SI-23}
\topic{Softwarový proces a jeho zlepšování: standardy, základní model SPI (software process improvement) a přístupy (prescriptive, inductive, best practices), proces organizace, projektu, jedince a souvislosti.}

\providecommand{\tightlist}{%
  \setlength{\itemsep}{0pt}\setlength{\parskip}{0pt}}

\begin{document}
\tableofcontents
\newpage

\section{Softwarový proces}
Množina aktivit potřebných k vývoji SW. Vždy je potřeba udělat:
\begin{itemize}
    \item specifikaci - co bude systém dělat
    \item architekrturu a design - jak a z čeho se systém bude skládat
    \item implementace - vlastní výroba systému
    \item validace - ověření, že systém dělá co má
    \item další vývoj - úpravy systému na základě (měnících se/dalších) požadavků
\end{itemize}

\subsection{Modely SW procesu}
\begin{itemize}
    \item plánovaný SW proces (plan-driven)
    \begin{itemize}
        \item waterfall
        \item iterative
    \end{itemize}
    \item agilní SW proces
    \begin{itemize}
        \item scrum
        \item XP
    \end{itemize}
\end{itemize}
\section{Zlepšování}
Provádí se pro zvýšení kvality softwaru, snížení nákladů nebo urychlit proces vývoje.
\newline
Zlepšení procesů = cyklický proces:
\begin{itemize}
    \item měření
    \item analýza
    \item změna
\end{itemize}
\subsection{PDCA model}
Jedná se o plán, který navrhuje jakým způsoveb řešit zlepšení/opravu systému.
\begin{itemize}
    \item plan
    \begin{itemize}
        \item identifikovat problém
        \item navrhnout řešení
        \item naplánovat změny
    \end{itemize}
    \item do - otestovat řešení
    \item check - zhodnotit výslekdy testu
    \item act - rozpracovat konečné řešení
\end{itemize}
\section{Zralost procesu}
Zavedení optimálních metod SW inženýrství do procestu vývoje. Úroveň zralosti je odrážena použitím optimálních technických a manažerských postupů.
V jednotlivých krocích za sebou:
\begin{itemize}
    \item initial
    \item repeatable - disciplinovaný proces
    \item defined - konzistentní
    \item managed - kontrolovaný a udržovaný
    \item optimizing - zaměření se na zlepšení
\end{itemize}
Základní varianty se rozdělují na systematický/dlouhodobý přístup (ISO, CMM - perspektivní vs SEL/NASA - induktivní) a best practices.
\newline
Perspektivníé - popisují, jak by měl systém vypadat. Aplikují zpravidla změny, které se osvědčili v jiných organizacích nebo oprimální SW. procesy.
\newline
Induktivní - analýza současného stavu
\begin{itemize}
    \item zamezení minulých problémů
    \item opakování minulých úspěchů
\end{itemize}
Best practices - použití především osvědčených praktik (přímo použitelné strategie, techniky a praktiky).
\end{document}