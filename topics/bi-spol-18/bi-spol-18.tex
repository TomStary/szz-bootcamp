\documentclass{szzclass}
\usepackage[czech]{babel}
\usepackage[margin=3cm]{geometry}
\usepackage{url}

% \subject{PA1}
% \code{BI-SPOL-18}
% \topic{Datové typy v programovacích jazycích. Staticky a dynamicky alokované proměnné, spojové seznamy. Modulární programování, procedury a funkce, vstupní a výstupní parametry. Překladač, linker, debugger.}

\begin{document}
\section{Datové typy v programovacích jazycích.}
\begin{description}
  \item[Číselné datové typy] - signed/unsigned; různé velikosti; little/big endian
  \begin{itemize}
    \item char
    \item integer
    \item float
    \item double
  \end{itemize}
  \item[String] - pole charů
  \item[Boolean]
  \item[Class, struct] 
  \item[NULL, void] 
  \item[enum, typedef]
  \item[array]
  \item[Ukazatele, reference]
\end{description}

\section{Staticky a dynamicky alokované proměnné.}
\begin{description}
  \item[staticky] - definované při kompilaci, alokované na zásobníku
  \item[dynamicky] - alokované při běhu, programátor se musí o alokovanou paměť starat (v případě C, malloc, free).
\end{description}

\subsection{Spojové seznamy}
Dynamicky alokovaná struktura. Každý prvek má nějakou hodnotu a ukazatele na další prvky. Cyklický, jednosměrný, obousměrný.

\section{Modulární programování}
Jednotlivé části se dají vyměňovat. Oddělení zodpovědnosti, zlepšení udržovatelnosti.
\subsection{Procedury a funkce, vstupní a výstupní parametry.}
Procedury jsou vesměs funkce bez návratové hodnoty. Vstupní parametry mají svůj typ, název a mohou mít defaultní hodnotu. Hodnoty parametrů se při volání funkce kopírují.

\section{Překladač}
\begin{itemize}
\item kompilátor
\item slouží k překladu vyššího jazyka do jazyka nižšího
\item nejčastěji překládá zdrojový kód do strojového kódu
\item vzniká strojový kód s neřešenými referencemi mezi moduly – objektový soubor
\item Skladá se ze dvou častí:
  \begin{itemize}
    \item Front-end  – parsuje zdrojový kód do vnítřní reprezentace kompilátoru. Tato část závisí na programovácím jazyce.
    \item Back-end – překladá vnítřní reprezentace do strojového kódu. Tato část zavísí na cílové architektuře (Intel, AVR, atd.).
  \end{itemize}
\end{itemize}
\subsection{Linker}
\begin{itemize}
\item řeší reference mezi objektovými soubory a knihovnami
\item slouži k propojení zkompilovaných modulů
\item => slouží k sestavení samostatně přeložených modulů a knihoven do funkčního celku
\item jeho výstupem je spustitelný soubor
\end{itemize}
\subsection{Debugger.}
\begin{itemize}
  \item nástroj pro ladění kódu a hledání chyb v programu
  \item používá se pro usnadnění pochopení, jak program funguje
\end{itemize}

\url{http://www.fit.vutbr.cz/~martinek/clang/gcc.html.}
\end{document}
