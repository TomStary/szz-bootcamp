\documentclass{szzclass}

\title{Lambda kalkul: definice pojmů, operací, reprezentace čísel.}

\begin{document}
\maketitle

\todo{Reorder to make more sense}

\section{Definice}

Lambda ($\lambda$) kalkul je formální systém používaný v teoretické informatice a matematice pro studium funkcí a rekurze, lambda kalkul je Turing-complete.
Funkce zapsané v lambda kalkulu lze poté vyhodnotit pomocí substituce. 

\subsection{Syntaxe lambda kalkulu}

U lambda kalkulu se používá prefixový zápis, tedy operátory se píší před operandy, například (+(* 5 3)(* 5 3)).
Vyhodnocení poté probíhá zleva doprava a to následovně:
\begin{enumerate}
    \item (+ 30 (* 5 3))
    \item (+ 30 30)
    \item 60
\end{enumerate}

Zápis funkce s proměnnou se poté zapisuje takto: $(\lambda . + x\ 1)$.

\subsection{Beta redukce}

Vyhodnocování funkcí v lambda kalkulu se dělá pomocí beta ($\beta$) redukce.
Beta redukce provedená na příkladu zápisu funkce:
\begin{enumerate}
    \item $(\lambda . + x\ 1)$
    \item $(\lambda . + x\ 1)2$
          2 za závorkou je argumentem fuknce.
    \item $(+ 2 1)$
    \item 3
\end{enumerate}

Další platné zápisy: $(\lambda . + x\ x) 2 => (+\ 2\ 2)$.

\subsection{Volné a vázáné proměnné}

V lambda kalkulu rozlišuje dva typy proměnných a to vázané a volné. Vázané proměnné jsou takové proměnné, které jsou zároveň argumentem
dané funkce. Lambda kalkul má lokální rozsah platnostnosti (scope).
Například pro fuknci $(\lambda x . + x\ y)$ je vázanou proměnnou x a y je v tomto případě volná proměnná.



\end{document}
