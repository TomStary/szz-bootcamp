\documentclass{article}
\usepackage{dependencies/szz-math}

\begin{document}

\section{Úvod}
Toto je pouze ukázka matematiky Latexu. Upozorňuji, že rovnice nemusí dávat smysl.

\section{Test}
$$
  x_1,\dots, x_n \in\mathbb{R}
$$
$$
  \bigO(n^2)=3n+2^n
$$

\section{Zarovnané rovnice}
\begin{align*}
x & \equiv 5 \pmod{6} \\
x & \equiv 3 \pmod{10} \describe{i s poznámkou} \\
x & \equiv 8 \pmod{15}
\end{align*}

\begin{center}
$x \equiv 5~(\text{mod } 6)$\linebreak
$x \equiv 3~(\text{mod } 10)$\linebreak
$x \equiv 8~(\text{mod } 15)$
\end{center}


\section{Delší výpočty}
Svůj výsledek ověřte matematickou indukcí.
\begin{equation*}
\begin{split}
s_n = & 1\cdot2 + 3\cdot4 + 5\cdot6 + \cdots + (2n-1) \cdot 2n = \\
 = & \sum\limits_{i=1}^{n} (2i-1)\cdot 2i =
     \sum\limits_{i=1}^{n} (4i^2-2i) =
     \sum\limits_{i=1}^{n} 4i^2 - \sum\limits_{i=1}^{n} 2i =
     4\sum\limits_{i=1}^{n} i^2 - 2\sum\limits_{i=1}^{n} i = \\
 = & 4 \cdot \frac{n(n+1)(2n+1)}{6} - 2 \cdot \frac{n(n+1)}{2} =
     n(n+1) (2 \cdot \frac{2n+1}{3} - 1) =
     n(n+1) \frac{4n-1}{3} \\
 = & \frac{1}{3} \cdot n(n+1)(4n-1)
\end{split}
\end{equation*}

\step{Základní krok} Dokážeme platnost tvrzení pro $n = 1$:
$$
    \frac{1}{3} \cdot (1+1)(4-1) = \frac{1}{3} \cdot 2 \cdot 3 = 2.
$$
Tvrzení v základním kroku platí.

\step{Indukční krok} Dokážeme $V(n) \Rightarrow V(n+1)$ pro $n\geq1$:
\begin{align*}
s_{n+1} = & \underbrace{1\cdot2 + 3\cdot4 + 5\cdot6 + \cdots + (2n-1) \cdot 2n}_{s_n} + (2n+1) \cdot (2n+2) = & \\
 \IP{=} & \frac{1}{3} \cdot n(n+1)(4n-1) + (2n+1) \cdot (2n+2) = & \describe{vytkneme $3$} \\
      = & \frac{n(n+1)(4n-1) + 3 \cdot (2n+1) \cdot (2n+2)}{3} = & \describe{vytkneme $n+1$} \\
      = & \frac{(n+1)(n(4n-1) + 12n + 6)}{3} = & \describe{upravíme druhou část zlomku} \\
      = & \frac{(n+1)(4n^2 + 11n + 6)}{3} = & \\
      = & \frac{(n+1)(n+2)(4n+3)}{3}
\end{align*}
\step{Závěr} Vše je zřejmé. \qed

\end{document}